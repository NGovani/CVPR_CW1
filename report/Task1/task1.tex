\section{Image Collection}

\paragraph{Task 1} We were instructed to capture to sets of images. The subject of both sets was an object placed in a 3D calibration grid and the difference between the two sets was the type of transformation between the images in said set. The first set \textbf{FD}, has images which horizontally pivoted around the grid, whilst the second set \textbf{HG}, contained images taken from the same position but with varying zoom and rotation. See Appendix \ref{apx:fd} and Appendix \ref{apx:hg} for the \textbf{FD} and \textbf{HG} sets respectively.

\begin{figure}[ht]
\begin{center}
   \begin{subfigure}{0.49\linewidth}
   \centering
   \includegraphics[width=0.8\linewidth]{../Images/DSC_0036.JPG}
   \caption{FD Sample Image}
   \label{fd:subfig:1}
   \end{subfigure}
   \begin{subfigure}{0.49\linewidth}
   \centering
   \includegraphics[width=0.8\linewidth]{../Images/DSC_0048.JPG}
   \caption{FD Sample Image}
   \label{fd:subfig:2}
   \end{subfigure}
\newline
   \begin{subfigure}{0.49\linewidth}
   \centering
   \includegraphics[width=0.8\linewidth]{../Images/DSC_0072.JPG}
   \caption{HG Sample Image}
   \label{hg:subfig:1}
   \end{subfigure}
   \begin{subfigure}{0.49\linewidth}
   \centering
   \includegraphics[width=0.8\linewidth]{../Images/DSC_0073.JPG}
   \caption{HG Sample Image}
   \label{hg:subfig:2}
   \end{subfigure}
\end{center}
\label{fig:1}
\end{figure}